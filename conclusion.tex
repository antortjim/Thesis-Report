\chapter{Conclusion of the Thesis}
\label{chap:conclusion}

The pipelines presented in chapter \ref{chap:pipeline} offer output-rich and easy to deploy cost-free pieces of software. All of them have pros and cons, and their combined usage was found to provide the best results, underlining the importance of the development of programs implementing open data formats exchangeable across tools. More functionalities can be added, including \textit{de novo} and \ac{PTM} search, fractioned samples and missing data management. BayesQuant, the probabilistic framework for relative quantification, presented in chapter \ref{chap:model}, can be passed a peptide level dataset and provide estimates of the uncertainty beyond frequentist \ac{log2FC} estimates. It can be customized to expand its currently implemented model. The lack of success in the attempt to model the peptide bias using the aminoacidic sequence reveals the complexity of its origin. Recently developed tools  could be applied in future work to extract a feature-rich representation of the collected spectra that could be used to model this phenomenon with a Deep Learning approach. The results from chapter \ref{chap:benchmark} manifest that \ac{NZ} can at the same time decrease computational costs and access the state of the art in proteomics research by making use of the latest open-source software. In summary, this work grants exciting opportunities for developing workflows in the field.